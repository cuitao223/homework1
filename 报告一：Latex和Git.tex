\documentclass{article}

% 导入中文宏包
\usepackage{ctex}
\usepackage{array}
\usepackage{caption}
\usepackage{hyperref}
% 设置页面边距
\usepackage{geometry}
\usepackage{graphicx}
\geometry{a4paper, left=2.5cm, right=2.5cm, top=3cm, bottom=3cm}

% 设置标题、作者和日期
\title{Latex与Git}
\author{23020007011崔涛}

\begin{document}

% 生成标题、作者和日期
\maketitle

% 心得报告正文
\section{实验目的}
本次课程主要讲授了版本控制(Git)以及Latex文档编辑,通过对两者的学习来加强对两大便捷工具的使用

\section{介绍}
\subsection{两大工具的优点}
Git 是一个分布式版本控制系统,它允许使用者跟踪文件和目录的变化历史Git 使得多人协作变得更加容易,多个开发者可以在同一个项目上工作,并轻松地合并各自的更改,方便了多人协作。

LaTeX 是一个高质量的排版系统,适合生成科学和数学文档。Latex能够处理复杂的公式和表格,并自动处理文档的格式和布局,方便了文章排版。
\section{练习内容}
\subsection{Git学习样例10个}
1. git init 可以在当前目录下创建git仓库,第一次使用需要进行如下配置\newspace
git config [--global] user.name "Your Name"   \newspace
git config [--global] user.email "email@example.com"  \newspace

2. Git中有关文件提交的基础命令

\begin{itemize}
    \item \verb|git status|: 显示当前的仓库状态
    \item \verb|git add <filename>|: 添加文件到暂存区
    \item \verb|git commit|: 创建一个新的提交
    \item \verb|git log|: 显示历史日志
    \item \verb|git log --all --graph --decorate|: 可视化历史记录
    \item \verb|git diff <filename>|: 显示与暂存区文件的差异
    \item \verb|git diff <revision> <filename>|: 显示某个文件两个版本之间的差异
    \item \verb|git checkout <revision>|: 更新 HEAD 和目前的分支

\end{itemize}
3.git reset 进行版本回退\textbf{,可以指定退回某⼀次提交的版本} 
。其格式为:git reset [-soft|-mixed|-hard] [HEAD] \newline
\begin{itemize}
    \item --soft 参数对于⼯作区和暂存区的内容都不变,只是将版本库回退到某个指定版本。
    \item  --mixed 为默认选项,使⽤时可以不⽤带该参数。该参数将暂存区的内容退回为指定提交版本内容,⼯作区⽂件保持不变。 
    \item --hard 参数将暂存区与⼯作区都退回到指定版本。切记⼯作区有未提交的代码时不要⽤这个命令,因为⼯作区会回滚。 
\end{itemize}
4.可以通过\verb|git checkout -- [file]|让工作区的文件回到最近一次 \verb|add| 或 \verb|commit|时的状态。 

5.克隆 \href{https://github.com/missing-semester-cn/missing-semester-cn.github.io.git}{本课程网站的仓库}并将版本历史可视化并进行探索
找出 是谁最后修改了 \verb|README.md| 文件?查看最后一次修改 \verb|_config.yml| 文件中 \verb|collections:| 行时的提交信息是什么?
\begin{itemize}
    \item 使用git clone 加上仓库链接即可克隆仓库。
    \item git log --all --graph --decorate将版本可视化历史化并进行探索 
    \item 使用git log -1 README.md即可找出最后修改的人
    \item git blame \_config.yml | grep collections即可找出最后一次修改 \verb|_config.yml| 文件中 \verb|collections:| 行时的提交信息
\end{itemize}
6.用Git时的一个常见错误是提交本不应该由Git管理的大文件,或者将含有敏感信息的文件提交给Git。尝试向仓库中添加一个文件并添加提交信息,然后将其从历史中删除
\begin{itemize}
    \item 输入(echo "password123">my\_password)(git add .)(git commit -m "add password123 to file")
 (git log HEAD)这四条指令来提交一些敏感信息
 \item 使用\verb|git filter-branch|删除记录,输入git filter-branch --force --index-filter\
 'git rm --cached --ignore-unmatch ./my\_password' \
 --prune-empty --tag-name-filter cat -- --all 即可
\end{itemize}

7.在 \~/.gitconfig 中创建一个别名,使在运行 git graph 时,您可以获得 git log –all –graph –decorate –oneline 的输出结果。\newline
解答:向文件中写入如下内容\newline
[alias]\newline
    graph = log --all --graph --decorate --oneline\newline
\newline

8.
配置全局 gitignore 文件来自动忽略系统或编辑器的临时文件 \newline
解答:通过执行 git config –global core.excludesfile \~/.gitignore\_global 在 \~/.gitignore\_global 中创建全局忽略规则,输入如下指令即可。git config --global core.excludesfile ~/.gitignore .Name

9.
git的分支操作

\begin{itemize}
    \item git branch branch\_name创建分支
    \item git branch查看分支
    \item git branch branch\_name  切换到指定分支上的最新版本
    \item git merge branch\_name合并分支


\end{itemize}
10.打开任意版本:git show hash(哈希值)

\subsection{Latex学习例子10个}
1.标题、日期与作者,这里可以用\verb|\title{}|、\verb|\author|、\verb|\date{}|来设置, 日期中选择\verb|\today|会 自动输出编译当天的日期。但是它们只能放在\verb|\documentclass|与\verb|\begin{document}|之间的区域),然后通过\verb|\maketitle|设来展现出来。

 
2.目录的实现:需要加入\verb|\tableofcontents|

3.插入图片需要使用\verb|graphicx|宏包 ,需要使用\verb|\includegraphics[width=8cm]{名字}|来实现\newline
4.无序列表的插入使用\verb|\begin{itemize}|和
\verb|\item|和\verb|\end{itemize}|即可|\newline
5.有序列表使用\verb|\begin{enumerate}| 和
\verb|\end{enumerate}|即可|\newline
6.插入代码:可以调用\verb|listings|宏包,并且在正文中使用lstlisting指令框住所想插入的代码 |\newline
\verb|\begin{listing}[language=python]|\newline
代码\newline
\verb|\end{listing}|\newline


7.参考文献使用natbib 包\newline
\verb|\usepackage{natbib}|\newline
参考\newline
\verb|\begin{document}|\newline
8.Latex中空格的添加\newline
\verb|\enspace|\newline
\verb|quad|\newline
9. 使用geometry宏包可以设置页面参数,如下所示\newline
\[\textbackslash usepackage[top = 30mm, bottom = 30mm, inner = 20mm, outer = 45mm]\{geometry\}\]

10.控制首行缩进。首先引用宏包
\[\textbackslash usepackage\{indentfirst\}\]之后使用
\[\textbackslash setlength\{\textbackslash parindent\}\{2em\}\]控制下文首行缩进两字符。


\section{解题感悟收获}
 通过学习LaTeX,我学会了如何进行高效排版,学会了常见的文章结构布局的实现方式,可以随心所欲控制文章格式,还学会了使用模板来写作,大大提高了我的论文写作效率,Latex还提高了我的文档排版的正确度,学会了利用Latex自动为题注进行编号并在文档中进行正确的引用,减少了手动编号可能导致的错误。\newline
 通过学习Git,我学会了如何进行版本控制,掌握了Git仓库的一些基本操作,这大大提高了我的团队协作效率,版本控制也不再是问题,我还学会了如何在大型项目中利用Git来变更追踪和管理,为以后的工作打下基础。
github路径
您可以在此查看项目的源代码: 

\url{https://github.com/cuitao223/homework1}
\end{document}
